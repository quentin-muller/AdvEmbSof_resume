\documentclass[resume]{subfiles}

\begin{document}
\section{Multi tâche}

Besoin de découper le code en petit élement 
\begin{itemize}
\item code lisible, structuré et documenté
\item code testable et modulaire
\item développement réutilise du code existant pour réduire le temps de développement
\item code peut être découper entre plusieurs personnes
\item code malléable et facile à mettre à jour 
\end{itemize}

Organisation des codes sous formes de classes et méthodes. Le c++ est donc parfait pour ceci.

Il y a deux types de tâches 
\begin{itemize}
\item event-triggered : se lance lors d'évènement
\item time-triggered : cyclique
\end{itemize}

\subsection{multitâche et OS}

C'est un OS qui nous permet de faire du multitâche 

\subsection{Tâches vs Processus}

Process :
\begin{itemize}
\item possède des ressources privée
\item un processus peut contenir plusieurs tâches 
\item nécessite plusieurs cœur (vrai //)
\end{itemize}

Task :
\begin{itemize}
\item un seul cœur (faux //)
\item ressource partagée entre deux taches possibles (gestion avec mutex ou sémaphore) 
\item s'exécute à tour de rôle selon leurs priorités
\end{itemize}

\subsection{RTOS}

Le temps réel garanti la complétion d'un process dans un intervalle de temps défini. Par contre il ne dit pas quel temps ça doit prendre, ça doit juste être toujours prédictible.

Un RTOS est un OS qui :
\begin{itemize}
\item Sert des applications temps-réelles
\item Répond à des requêtes dans une délai garantit et prédictible
\item Process les données dans un nombre déterministe de cycles
\end{itemize}

Un RTOS est plus pour des performances déterministiques, plutôt que un high troughput
\begin{itemize}
\item Les performances peuvent être mesurées par latence (temps qu'il prend)
\item Les performances peuvent être mesurées par gigue (variation du temps qu'il prend)
\item Soft RTOS : plus de gigue, généralement répond aux deadlines
\item Hard RTOS : moins de gigue, répond aux deadlines de manière déterministe
\end{itemize}

\subsection{Tâches et RTOS}

C'est l'OS qui détermine quelle section du programme va être exécuter et l'accès aux ressources Il gère aussi la synchronisation entre les tâches.

\subsection{Mbed OS}

\subsubsection{Task scheduling}

mbed utilise RTX5 pour ARM avec une faible latence

Les priorités sont configurées telles qu'aucune préemption apparait entre ces handler

C'est une combinaison de priorité et d'un scheduling de round-robin
\begin{itemize}
\item Round Robin pour les tâches de même priorités
\item Basé sur la priorité pour les autres tâches
\end{itemize}

\subsubsection{Scheduling Options}
\begin{itemize}
\item Preemptive : chaque tâche à une priorité différente
\item Round-Robin : toutes les tâches ont la même priorité
\item Co-operative multi-tasking : toutes les tâches ont la même priorité et chaque tâche va jusqu'à un appelle bloquant
\end{itemize}

Par défaut c'est Round-Robin préemptif

\subsubsection{Context switching}

Le Thread Control Block rend le changement de contexte plus facile

Cela prend du temps (200-300 cycles)
\begin{itemize}
\item save state, load other state
\item change state
\end{itemize}

\subsubsection{Thread States}
\begin{itemize}
\item Running
\item Ready
\item Waiting/Blocked
\item Inactive/Terminated
\end{itemize}

\subsubsection{Thread synchronization}

Dans un système multi-tâches, les différentes tâches peuvent tenter d'obtenir la même ressource ou peuvent attendre l'apparition de différents évènements

Dans certains cas, un tâche donnée peut entrer dans un état Waiting ou Blocked

Il y a plusieurs Waiting states

\subsection{Evènements}

Les évents sont utiles pour attendre certaines conditions
\begin{itemize}
\item Un thread attends pour une condition spécifique 
  \item La condition peut être faite d'un ou plusieurs flags (AND/OR)
  \item Attente avec timeout est possible

\item Un autre thread set cette condition spécifique
\item Pas d'attente active
\end{itemize}

\subsection{Mutex}

Protection des ressources partagées
\begin{itemize}
\item Ils s'assurent qu'un seul thread peut avoir accès à une section de code critique
\item On doit être attentif au deadlock ou starvation
  \item Faire attention quand on a plusieurs mutex
  \item Toujours redonner le mutex après utilisation
\end{itemize}

\subsection{Deadlock}

Condition pour un deadlock (si toutes vraies)
\begin{itemize}
\item Le Mutex est inévitable
\item Préemption est inévitable
\item Prévenir les conditions circulaire d'attente
\end{itemize}

Si deadlock alors reset (watchdog)

\subsection{Sémaphore}

Producteur / Consommateur

Un sémaphore gère l'accès des threads à un set de ressources partagées d'un certain type
\begin{itemize}
\item A la différence des mutex, un sémaphore peut contrôler l'accès à plusieurs ressources partagées
\item Une sémaphore active l'accès et la gestion d'un groupe de périphériques identiques
\end{itemize}

\subsection{Queue/mail}
\begin{itemize}
\item Queue : FIFO
\item Mail : comme une Queue avec de la mémoire pour des messages
\end{itemize}

\subsection{Inversion de priorité}

Les tâches les plus hautes bloquent les tâches plus basses cela modifie le temps de réponse

Pour empêcher cela il faut suivre les Resources Access Protocols

\subsubsection{Principales sources de d'inversion de priorité}
\begin{itemize}
\item section non-préemptive
\item ressources partagées
\item synchronisation et exclusion mutuel
\end{itemize}

\subsection{Resources Access Protocols}
\begin{itemize}
\item Inversion de priorité est bornée
\item Evite les deadlock
\item Evite les blocages non-nécessaire
\item Calcul facile du temps max de blocage
\item Nombre de lbocage max
\item Facile à implémenter
\end{itemize}

\subsubsection{Types}
\begin{itemize}
\item Non preemption (NPP) : haute pri si lock mutex ok
\item Priotity Inhertiance (PIP) : si bloque tâche plus haute pri alors prend sa pri (MBED)
\item High Locker (HLP) : prend la pri de du mutex. La pri du mutex est definit par la plus haute qui l'utilise
\item Priority Ceiling (PCP) : PIP et HLP prend le plus haut
\end{itemize}

\subsubsection{Résumé}

\begin{table}[H]
\centering
    \begin{tabular}{|c|c|c|c|c|}
    \hline
                       & NPP  & PIP  & HLP    & PCP  \\
    \hline
	Inv priorité       & yes  & yes  & yes    & yes  \\
	\hline
	Evite deadlock     & yes  & no   & yes    & yes  \\
	\hline
	Evite blocage      & no   & yes  & yes/no & yes  \\
	\hline
	calcul max blocage & easy & hard & easy   & easy \\  
    \hline
    \end{tabular}
\end{table}



\end{document}